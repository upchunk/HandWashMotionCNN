\chapter{PENUTUP}
\label{chap:penutup}

% Ubah bagian-bagian berikut dengan isi dari penutup

\section{Kesimpulan}
\label{sec:kesimpulan}

Berdasarkan hasil pengujian dan analisis yang telah dilakukan pada BAB 4, didapat kesimpulan sebagai berikut sebagai berikut:

\begin{enumerate}[nolistsep]

  \item Pada Pengujian berdasarkan sudut pengambilan video, Apabila sudut terlampau jauh dari sudut pengambilan video pada dataset, maka akurasi sistem menurun

  \item Pada Pengujian berdasarkan setup pengambilan video, Apabila setup yang digunakan menyebabkan perubahan bentuk pada citra, maka akurasi sistem akan menurun

  \item Dataset yang dimiliki masih dirasa kurang untuk mendapat model klasifikasi gerakan cuci tangan yang lebih general / dapat digunakan pada skenario yang lebih luas

\end{enumerate}

\section{Saran}
\label{chap:saran}

Untuk pengembangan lebih lanjut pada Sistem Klasifikasi Gerakan Mencuci Tangan ini, dapat dilakukan berberapa hal antara lain:

\begin{enumerate}[nolistsep]

  \item Untuk menggunakan sistem ini secara realtime, dapat dilakukan pemilihan model yang lebih cepat dan pemilihan metode input dan output yang lebih efisien.
  
  \item Untuk meningkatkan efisiensi training, dapat dilakukan optimasi kode program dan \textit{Hyperparameter} yang lebih mendalam

  \item Untuk meningkatkan efisiensi training, dapat pula sistem dapat dikembangkan sebuah metode untuk mengekstrak bagian tangan dan memisahkannya dari background ke dalam sistem klasifikasi ini

  \item Untuk meningkatkan kememampuan \textit{generalization} dari sistem klasifikasi ini, dapat dilakukan dengan penambahan jumlah Training Dataset dari sumber dan setup pengambilan yang lebih beragam, sehingga sistem lebih leluasa digunakan pada situasi dan setup yang berbeda

\end{enumerate}
