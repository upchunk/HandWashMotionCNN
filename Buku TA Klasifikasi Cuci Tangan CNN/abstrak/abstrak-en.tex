\begin{center}
  \large\textbf{ABSTRACT}
\end{center}

\addcontentsline{toc}{chapter}{ABSTRACT}

\vspace{2ex}

\begingroup
  % Menghilangkan padding
  \setlength{\tabcolsep}{0pt}

  \noindent
  \begin{tabularx}{\textwidth}{l >{\centering}m{3em} X}
    % Ubah kalimat berikut dengan nama mahasiswa
    \emph{Name}     &:&	Habibul Rahman Qalbi\\

    % Ubah kalimat berikut dengan judul tugas akhir dalam Bahasa Inggris
    \emph{Title}    &:& \emph{CNN} Based Hand Washing Motion Classification \\

    % Ubah kalimat-kalimat berikut dengan nama-nama dosen pembimbing
    \emph{Advisors} &:& 1. Dr. Eko Mulyanto Yuniarno, S.T., M.T. \\
    				& & 2. Reza Fuad Rachmadi, S.T., M.T., Ph.D \\
  \end{tabularx}
\endgroup

% Ubah paragraf berikut dengan abstrak dari tugas akhir dalam Bahasa Inggris
\emph{Abstract—Hand Washing is the first step when it comes to maintain health and hygine, by doing so we prevent the spread of deseases. Even though, there is still many people that didn’t wash their hand properly. By using Deep Learning algorithm, we could know if someone had wash their hand properly by capturing their movement with a camera and processing the data through a CNN Model. We hope that this technology could help to monitor and ensure that people had wash their hand properly, specially on public places where the potential spread of diseases is quite high.}

% Ubah kata-kata berikut dengan kata kunci dari tugas akhir dalam Bahasa Inggris
\emph{Keywords}: \emph{Hand}, \emph{Health}, \emph{Classification}, \emph{CNN}.
