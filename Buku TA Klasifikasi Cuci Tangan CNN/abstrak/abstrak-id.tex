\begin{center}
  \large\textbf{ABSTRAK}
\end{center}

\addcontentsline{toc}{chapter}{ABSTRAK}

\vspace{2ex}

\begingroup
  % Menghilangkan padding
  \setlength{\tabcolsep}{0pt}

  \noindent
  \begin{tabularx}{\textwidth}{l >{\centering}m{2em} X}
    % Ubah kalimat berikut dengan nama mahasiswa
    Nama Mahasiswa    &:& 	Habibul Rahman Qalbi\\

    % Ubah kalimat berikut dengan judul tugas akhir
    Judul Tugas Akhir &:&	Klasifikasi Gerakan Mencuci Tangan  \\
    				  & &	Berbasis \emph{CNN} \\

    % Ubah kalimat-kalimat berikut dengan nama-nama dosen pembimbing
    Pembimbing        &:& 1. Dr. Eko Mulyanto Yuniarno, S.T., M.T. \\
                      & & 2. Reza Fuad Rachmadi, S.T., M.T., Ph.D \\
  \end{tabularx}
\endgroup

% Ubah paragraf berikut dengan abstrak dari tugas akhir
Cuci tangan merupakan langkah awal dalam kesehatan, dengan mencuci tangan kita dapat mencegah penyebaran penyakit. Akan tetapi, masih banyak masyarakat yang tidak sadar akan tata cara mencuci tangan yang baik sehingga tidak bersih sepenuhnya. Pemanfaatan teknologi Deep Learning dapat menjadi solusi untuk mengetahui apakah masyarakat telah mencuci tangan dengan benar, menggunakan kamera sebagai input yang kemudian di proses menggunakan Algoritma \emph{Convolutional Neural Network (CNN)} kita dapat mengklasifikasikan gerakan - gerakan yang dilakukan pengguna saat mencuci tangan. Harapannya hasil penelitian ini dapat membantu dalam memantau dan memastikan apakah masyarakat mencuci tangan dengan benar.

% Ubah kata-kata berikut dengan kata kunci dari tugas akhir
Kata Kunci: Tangan, Kesehatan, Klasifikasi, \emph{CNN}.
